\documentclass{article}

\title{Secure Pairing Methods for Mobile Devices}
\date{\today}
\author{Erik Boss, Max Hovens, Anton Jongsma, Aram Verstegen}

\begin{document}
\begin{abstract}
\end{abstract}

\section{Introduction}

Device \textit{pairing} is the process of connecting two devices with mutual authentication which does not rely on an external trust anchor as is the case with traditional authentication.
Mobile devices are an interesting field for research and development of secure pairing methods because they have relevant applications in practice.

The focus of this research would be to investigate secure pairing methods for mobile devices in general.
Furthermore, we shall endeavour to deliver a practical implementation of a secure pairing method on the Android platform.

\section{Research Themes}
Our preliminary literature study has suggested some interesting pairing methods, such as using GPS data to agree upon a route travelled as a shared secret; % TODO reference
using front-facing camera's to have devices read each other's screens~\cite{saxena2006secure};
using the balance sensors in two devices which are shaken about together in order to use the measurements as an agreed upon secret~\cite{mayrhofer2009shake}.

\section{Approach for Practical Implementation}

We have chosen the Google Android mobile device platform as it is accessible to new developers, more so than the Apple iOS platform.
Coincidentally, Android devices are much more prevalent than iOS devices.

\section{Literature}

\bibliographystyle{plain}
\bibliography{../references/references.bib}

\end{document}

% vim:tw=0:wrap
