\documentclass{article}

\title{Secure Pairing Methods for Mobile Devices}
\date{\today}
\author{Erik Boss, Max Hovens, Anton Jongsma, Aram Verstegen}

\begin{document}
%\begin{abstract}
%\end{abstract}

\section{Introduction}

Device \textit{pairing} is the process of connecting two devices who have never seen eachother before with mutual authentication which does not rely on an external trust anchor as is the case with traditional authentication.
Mobile devices are an interesting field for research and development of secure pairing methods because they have relevant applications in practice.

The focus of this research would be to investigate secure pairing methods for mobile devices in general.
Furthermore, we shall endeavour to deliver a practical implementation of a secure pairing method on the Android platform.



\section{Research Themes}
The focus of the theoretical part of this research is to provide a broad in-depth overview of different pairing methods as described in the the literature. Our preliminary literature study has suggested some interesting pairing methods, such as using GPS data to agree upon a route travelled as a shared secret; % TODO reference
using front-facing camera's to have devices read each other's screens~\cite{saxena2006secure},
using the balance sensors in two devices which are shaken about together in order to use the measurements as an agreed upon secret~\cite{mayrhofer2009shake}, and using sound~\cite{soriente2008hapadep}.
Furthermore we found several comparative surveys of pairing methods~\cite{kumar2009comparative}\cite{kobsa2009serial}\cite{uzun2007usability}.


\section{Practical Implementation}

For the practical part of this research we are going to implement a secure pairing method over the visual channel. By displaying a QR code on the screen of the first device, and scanning it on the second device, we can transfer quite a large amount of information between the devices. This wil be the basis of the paing method.

We have chosen the Google Android mobile device platform to implement this method.
It is more accessible to new developers than the Apple iOS platform, and we already have some experience writing software for the Android platform.
Coincidentally, Android devices are much more prevalent than iOS devices.

\section{Literature}

\bibliographystyle{plain}
\bibliography{../references/references.bib}

\end{document}

% vim:tw=0:wrap
