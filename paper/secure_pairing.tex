\documentclass[conference, 11pt]{sty/IEEEtran}

\title{Secure Pairing}
\author{Erik Boss \and Max Hovens \and Anton Jongsma \and Aram Verstegen}
\date{\today}

\usepackage{hyperref}

\begin{document}

\maketitle

\begin{abstract}
    
\end{abstract}

\section{Introduction}
\label{sec:introduction}


Device \textit{pairing} is the process of connecting two devices who have never seen each other before with mutual authentication which does not rely on an external trust anchor as is the case with traditional authentication.
Accomplishing this in a correct and secure way in a mobile setting is an interesting problem.
In particular, the obvious increase in devices using wireless signals makes the problem more relevant as it is often necessary to have these devices communicate securely with each other.
How do we make sure that the devices know that they are communicating with the correct parties?
How do we agree on keys?
These problems are only exacerbated by the fact that often we are dealing with very constrained devices in terms of size and performance.
Even if we do not necessarily have problems in terms of resources we can still have problems in terms of usability.
For instance, on modern phones we have a great deal of resources in terms of performance.
The dependence on, say, a PIN that needs to be remembered greatly hampers the technique from a usability point of view. 
Passwords or PINs are forgotten all the time.

In this paper we shall do a number of things.
First, we shall provide a survey of techniques pertaining to what appears to be the state-of-the-art in secure pairing.
We shall then proceed to briefly assess the state-of-the-art from both a security and a usability perspective.
Furthermore, we shall strive to implement a secure pairing system between two mobile phones relying on QR-codes to set up a secure connection.
This can serve as a convenient platform for discussing the security and usability of such systems, in particular the trade-offs between those two concepts.

\section{Pairing methods}

In this section we describe several pairing methods proposed in the literature.

\subsection{Image comparing}
An early approach to the problem of secure pairing involves image compairing.
On both devices an image of the OOB data is shown and the user is asked to compare both images.
Examples are \textit{Snowflake}\cite{goldberg1996visual}, \textit{Random Arts Visual Hash}\cite{perrig1999hash} and \textit{Colorful Flag}\cite{dohrmann2002public}.
These methods require both devices to have a display with a reasonable resolution, making it impractical for devices like bluetooth headsets. However, all modern smartphones have such a display.

\subsection{Seeing is believing}
In \cite{mccune2005seeing} camera phones use the \textit{visual channel} to transfer some information. In this method one device uses its camera to take a snapshot of a barcode encoding for example the public key of the other device. The minimum for this approach to work is one device with a display, and one with a camera, making it again unsuitable for low-end devices.

\subsection{Blinking lights}
Related to \textit{Seeing is believing} is the \textit{Blinking lights} method.
Instead of a display and a camera the devices in the method use a LED and a light-sensor or camera.
The LED-equipped device transmits the authentication data by blinking the LED.
The other device recovers the data by recording the blinking and looking at the inter-blink gaps.

\subsection{Shake well before use}
A different approach is taken in \cite{mayrhofer2009shake}, where two devices are supposed to be shaken together to do the pairing.
It matches features extracted from the sensor data to derive a cryptographic key.

\section{Usability study}

\section{Security study}

\section{Implemenation}

\section{Conclusion}
\label{sec:conclusion}

\bibliographystyle{plain}
\bibliography{../references/references.bib}

\end{document}

% vim:tw=0:wrap
