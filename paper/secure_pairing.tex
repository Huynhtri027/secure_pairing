\documentclass[conference, 11pt]{sty/IEEEtran}

\title{Secure Pairing}
\author{Erik Boss \and Max Hovens \and Anton Jongsma \and Aram Verstegen}
\date{\today}

\usepackage{hyperref}

\begin{document}

\maketitle

\begin{abstract}
    TODO: Will be written after everything is finished.
\end{abstract}

\section{Introduction}
\label{sec:introduction}

The proliferation of devices, mobile or otherwise, communicating over wireless channels leads to interesting problems.
In particular, the ease with which an adversary can eavesdrop said communication is rather astounding and thus a risk from a security point of view.
The obvious solution is to, of course, attempt to secure this wireless channel.
This is where \emph{secure device pairing} comes in.

Secure device pairing (or secure pairing, for short) is the process of creating a secure channel between two devices that have no previous security context.
That is, a secure pairing scheme needs to be able to provide mutual authentication without relying on external trust anchors.
Because we cannot rely on an external infrastructure as a basis of trust most of the standard techniques are not easily applicable.
In particular, there is an often cited need to do some of the communication over an auxiliary channel, i.e., so-called out-of-band (OOB) channels.

These OOB channels are managed by the users themselves and typically rely on human interaction to authenticate the communication over the wireless channel.
Suppose we use human visual capabilities as an OOB channel.
The idea is that this channel is very hard to interfere with for an attacker without the human user(s) noticing.
Overall, there are many such OOB channels, some of which we shall discuss in subsequent sections in the paper.
One problem with such channels is that they often rely on a certain common set of interfaces.
However, interface-wise almost no two devices are alike and it is common for two very different devices to need a secure communication channel.
For instance, a recent mobile phone might have a (virtual) keyboard, a camera and a smattering of sensors.
A simple wireless bluetooth headset, however, likely has almost none of these things and is probably limited to just communicating over a wireless channel.

There are two elements to a secure pairing scheme.
In essence, we have the content and the method of pairing.
The content entails the messages sent, i.e., the protocol; the method entails the way in which certain interfaces are used in order to use the protocol.
Most of our focus will be on the latter, since the protocols are typically not very complex.
We will, however, mention some of the ways in which the information can be sent.

We structure our paper as follows.
First, we shall provide a survey of techniques pertaining to what appears to be the state-of-the-art in secure pairing with regards to both the content and the method of pairing.
We shall then proceed to briefly assess the state-of-the-art in secure device pairing from both a security and a usability perspective.
Furthermore, we shall strive to implement a secure pairing system between two mobile phones relying on QR-codes to set up a secure connection.
This serves as a case study and a convenient platform for discussing the security and usability of such systems, in particular the trade-offs between those two concepts.

\section{Cryptographic protocols for secure device pairing}
\label{sec:cryptographic_protocols_for_secure_device_pairing}

In this section we shall discuss a few general ways of accomplishing the authentication of two devices assuming a suitable OOB channel is in place.
The idea is that these protocols can determine the content of the information being exchanged without relying too much on the particular channel.
Again, this the separation between the content and the method (or representation) of the secure pairing technique.
Note, however, that this does not mean that every, or even most, of the techniques described in the next section use one of these protocols.
Many of those exploit particular characteristics of their respective representations or OOB channels.

\subsection{Talking to Strangers}
\label{ssec:talking_to_strangers}

A fairly old method, it describes a system~\cite{balfanz2002talking} that relies on a suitable OOB channel in order to transmit pre-authentication data after which standard protocols can be implemented.
Specifically, each party sends their public key over the normal wireless channel and a hash of said public key over the OOB channel.

More formally, the pre-authentication looks as follows given addresses $addr_A$ and $addr_B$ for parties $A$ and $B$, their respective public keys $pk_A$ and $pk_B$ and a collision resistant hash function $h$:

\begin{enumerate}
    \item $A \rightarrow B: addr_A, h(pk_A)$
    \item $B \rightarrow A: addr_B, h(pk_B)$
\end{enumerate}

After which we can proceed with a standard key exchange protocol.

The initial proposal used infra-red signals as an OOB channel.
Note, however, Infra-red as an OOB channel may not be as secure as one would like~\cite{kumar2009comparative}.
The overall technique is broad enough to be useful for other OOB channels.


\section{Secure device pairing methods}
\label{sec:secure_device_pairing_methods}

In this section we describe several pairing methods proposed in the literature.

\subsection{Image comparison}
An early approach to the problem of secure pairing involves image comparison.
On both devices an image of the OOB data is shown and the user is asked to compare both images.
Examples are \textit{Snowflake}\cite{goldberg1996visual}, \textit{Random Arts Visual Hash}\cite{perrig1999hash} and \textit{Colorful Flag}\cite{dohrmann2002public}.
These methods require both devices to have a display with a reasonable resolution, making it impractical for devices like bluetooth headsets. However, all modern smartphones have such a display.

\subsection{Seeing is believing}
In \cite{mccune2005seeing} camera phones use the \textit{visual channel} to transfer some information.
In this method one device uses its camera to take a snapshot of a barcode encoding for example the public key of the other device.
The minimum for this approach to work is one device with a display, and one with a camera, making it again unsuitable for low-end devices.

\subsection{Blinking lights}
Related to \textit{Seeing is believing} is the \textit{Blinking lights} method.
Instead of a display and a camera the devices in the method use a LED and a light-sensor or camera.
The LED-equipped device transmits the authentication data by blinking the LED.
The other device recovers the data by recording the blinking and looking at the inter-blink gaps.

\subsection{Shake well before use}
A different approach is taken in \cite{mayrhofer2009shake}, where two devices are supposed to be shaken together to do the pairing.
It matches features extracted from the sensor data to derive a cryptographic key.

\section{Assessment of the state-of-the-art}
\label{sec:assessment_of_the_state_of_the_art}

\subsection{Security assessment}
\label{ssec:security_assessment}

\subsection{Usability assessment}
\label{ssec:usability_assessment}

\section{Implementation}
\label{sec:implementation}

TODO: We do not have an implementation just yet.

\subsection{Overview}
\label{ssec:overview}

TODO: We need an implementation first.

\section{Conclusion}
\label{sec:conclusion}

TODO: To be done after the implementation section is finished.

\bibliographystyle{plain}
\bibliography{../references/references.bib}

\end{document}

% vim:tw=0:wrap
