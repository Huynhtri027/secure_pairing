\documentclass[conference, 11pt]{sty/IEEEtran}
% \documentclass[paper=a4, fontsize=10pt]{scrartcl}

\title{Secure Pairing}
\author{Erik Boss \and Max Hovens \and Anton Jongsma \and Aram Verstegen}
\date{\today}

\usepackage{hyperref}
\usepackage{amsmath, amssymb}

\begin{document}

\maketitle

\begin{abstract}
    TODO: Will be written after everything is finished. Such is the nature of abstracts.
\end{abstract}

\section{Introduction}
\label{sec:introduction}

The proliferation of devices, mobile or otherwise, communicating over wireless channels leads to interesting problems.
In particular, the ease with which an adversary can eavesdrop said communication is rather astounding and thus a risk from a security point of view.
The obvious solution is to, of course, attempt to secure this wireless channel.
This is where \emph{secure device pairing} comes in.

Secure device pairing (or secure pairing, for short) is the process of creating a secure channel between two devices that have no previous security context.
That is, a secure pairing scheme needs to be able to provide mutual authentication without relying on external trust anchors.
Because we cannot rely on an external infrastructure as a basis of trust most of the standard techniques are not easily applicable.
In particular, there is an often cited need to do some of the communication over an auxiliary channel, i.e., so-called out-of-band (OOB) channels.

These OOB channels are managed by the users themselves and typically rely on human interaction to authenticate the communication over the wireless channel.
Suppose we use human visual capabilities as an OOB channel.
The idea is that this channel is very hard to interfere with for an attacker without the human user(s) noticing.
Overall, there are many such OOB channels, some of which we shall discuss in subsequent sections in the paper.
One problem with such channels is that they often rely on a certain common set of interfaces.
However, interface-wise almost no two devices are alike and it is common for two very different devices to need a secure communication channel.
For instance, a recent mobile phone might have a (virtual) keyboard, a camera and a smattering of sensors.
A simple wireless bluetooth headset, however, likely has almost none of these things and is probably limited to just communicating over a wireless channel.

There are two elements to a secure pairing scheme.
In essence, we have the content and the method of pairing.
The content entails the messages sent, i.e., the protocol; the method entails the way in which certain interfaces are used in order to use the protocol.
Most of our focus will be on the latter, since the protocols are typically not very complex.
We will, however, mention some of the ways in which the information can be sent.

We structure our paper as follows.
First, we shall provide a survey of techniques pertaining to what appears to be the state-of-the-art in secure pairing with regards to both the content and the method of pairing.
We shall then proceed to briefly assess the state-of-the-art in secure device pairing from both a security and a usability perspective.
Furthermore, we shall strive to implement a secure pairing system between two mobile phones relying on QR-codes to set up a secure connection.
This serves as a case study and a convenient platform for discussing the security and usability of such systems, in particular the trade-offs between those two concepts.

\section{Cryptographic protocols for secure device pairing}
\label{sec:cryptographic_protocols_for_secure_device_pairing}

TODO: Add more details perhaps (maybe a picture?).

In this section we shall discuss a few general ways of accomplishing the authentication of two devices assuming a suitable OOB channel is in place.
The idea is that these protocols can determine the content of the information being exchanged without relying too much on the particular channel.
Again, this the separation between the content and the method (or representation) of the secure pairing technique.
Note, however, that this does not mean that every, or even most, of the techniques described in the next section use one of these protocols.
Many of those exploit particular characteristics of their respective representations or OOB channels.

\subsection{Talking to Strangers}
\label{ssec:talking_to_strangers}

A fairly old method, it describes a system~\cite{balfanz2002talking} that relies on a suitable OOB channel in order to transmit pre-authentication data after which standard protocols can be implemented.
Specifically, each party sends their public key over the normal wireless channel and a hash of said public key over the OOB channel.

More formally, the pre-authentication looks as follows given addresses $addr_A$ and $addr_B$ for parties $A$ and $B$, their respective public keys $pk_A$ and $pk_B$ and a collision resistant hash function $h$:

\begin{enumerate}
    \item $A \rightarrow B: addr_A, h(pk_A)$
    \item $B \rightarrow A: addr_B, h(pk_B)$
\end{enumerate}

After which we can proceed with a standard key exchange protocol.

The initial proposal used infra-red signals as an OOB channel.
Note, however, Infra-red as an OOB channel may not be as secure as one would like~\cite{kumar2009comparative}.
The overall technique is broad enough to be useful for other OOB channels.

\subsection{Short Authenticated Strings}
\label{ssec:short_authenticated_strings}

A \emph{Short Authenticated Strings} (SAS) protocol achieves mutual authentication, much like the previously discussed protocol.
The main advantage is that the amount of data necessary for transfer over the OOB channel is fairly small.

The protocol primarily utilizes commitment schemes in order to get an agreement on each other's public keys.
The scheme, as denoted in~\cite{saxena2008automated} and~\cite{saxena2006secure} works by having two random values.
One party, let us say Alice, commits to her random value which which, when opened by sending the decommit value, gives access to Alice's random value.
In the meanwhile the other party has sent its random value in the clear to Alice.
Both parties now communicate over an OOB channel in order to check if both parties have received the same random values.

\section{Secure device pairing methods}
\label{sec:secure_device_pairing_methods}

TODO: Extend the explanations and add a few more techniques.

In this section we describe several pairing methods proposed in the literature.

\subsection{Image comparison}
An early approach to the problem of secure pairing involves image comparison.
On both devices an image of the OOB data is shown and the user is asked to compare both images.
The underlying protocol is the one described in \autoref{ssec:talking_to_strangers}.
Examples are \textit{Snowflake}\cite{goldberg1996visual}, \textit{Random Arts Visual Hash}\cite{perrig1999hash} and \textit{Colorful Flag}\cite{dohrmann2002public}.
These methods require both devices to have a display with a reasonable resolution, making it impractical for devices like bluetooth headsets.
However, all modern smartphones have such a display and for that reason the technique is not without its uses.

\subsection{Seeing is Believing}
In \cite{mccune2005seeing} camera phones are used in order to use the \textit{visual channel} to transfer some information.
In this method one device uses its camera to take a snapshot of a barcode encoding of, e.g., the public key of the other device.
The minimum for this approach to work is one device with a display, and one with a camera, making it again unsuitable for low-end devices.

\subsection{Blinking lights}
Related to \textit{Seeing is Believing} is the \textit{Blinking lights} method.
Instead of a display and a camera the devices in the method use a LED and a light-sensor or camera.
The LED-equipped device transmits the authentication data by blinking the LED.
The other device recovers the data by recording the blinking and looking at the gaps between distinct blinks.

\subsection{Shake well before use}
A different approach is taken in \cite{mayrhofer2009shake}, where two devices are supposed to be shaken together in one hand to do the pairing.
This is actually a very clever OOB channel, there really seems to be no way to eavesdrop with any effect on such a scheme.
Also, shaking is a very distinctive gesture, making it relatively easy to derive measurements that can be mapped to bitstrings.
It matches features extracted from the sensor data to derive a cryptographic key.

It is also noted that while the shaking has sufficient entropy to be useful in key generation, its primary use is in key verification.
That is, similar to the SAS protocol in \autoref{ssec:short_authenticated_strings} it may be more useful to just send regular randomly bytes to each other and \emph{then} verify that was done correctly by means of an OOB channel, in this case the shaking gesture.

The feature extraction is a process that involves corrections for time alignment and spatial alignment before the data is usable at all.
Indeed, it is easy to see that the particular way one holds both devices before shaking can lead to somewhat different measurements in each device.
In particular, the time alignment is done through segmentation and the spatial alignment through normalization.

\subsection{Loud and Clear}
\label{ssec:loud_and_clear}

Loud and clear~\cite{goodrich2006loud} is a technique that uses audio as an OOB channel.
The underlying protocol is the same as in \autoref{ssec:talking_to_strangers}.
It works by vocalizing, using a specialized text-to-speech engine, authentication strings into syntactically correct English sentences.

Two modes are possible, speaker-speaker mode and speaker-display mode.
They require the user to respectively compare two vocalized strings or compare the vocalized string with a textual representation of such a string.

\subsection{HAPADEP}
\label{ssec:hapadep}

Similar to Loud-and-Clear, HAPADEP~\cite{soriente2008hapadep} utilizes sound in order to get a pairing.
Its intended use is in situations where one may not have an available wireless connection.
It can use the audio channel for all messages instead.
That said, it can also be used in such a way that only some messages are sent using this audio channel, everything else can go via wireless.


\subsection{Graphical passwords}
\label{ssec:graphical_passwords}

Graphical passwords are somewhat different from what we have seen so far.
As the name implies it entails the entering of passwords in a graphical manner as opposed to the usual practice of entering PINs or alphanumeric passwords in general.
The idea is that these graphical passwords can provide a similar level of security while being more memorable.
Do note that these techniques do not necessarily use an OOB channel, they really are passwords in the more traditional sense, which is really what sets them apart from some of the other techniques discussed in this paper.

In \cite{gajos2006exploring} five different types of graphical passwords are discussed: \emph{Pass-Go}, \emph{UYI}, \emph{TAPI}, \emph{CCP} and \emph{MIBA}.
As an addition you might consider the Android gesture swipe unlocking scheme as a more recent example of such a system.
However, that particular scheme suffers from problems in terms of predictability of patterns, lack of possibilities, easy should surfing and even smudge attacks~\cite{aviv2010smudge}.
Whatever the case, the five techniques mentioned in \cite{gajos2006exploring} seem to cover quite a few different types of graphical passwords and we shall reproduce some of the descriptions with their respective characteristics here.

\subsubsection{Pass-Go}
\label{sssec:pass_go}

Pass-Go entails the drawing dots and lines on a $n$-by-$n$ grid.
The grid can be colored in such a way that the user is being given hints pertaining to the correct password.
The size of $n$ primarily decides the security of the password, although $n$ does not really have to be very large.
$n = 5$ was suggested for phones but since screen sizes have increased we believe that this can be increased proportionally.
Also, the security can be enhanced by coloring the lines and dots differently.

\subsubsection{UYI}
\label{sssec:uyi}

UYI is a simple scheme where the user has to pick a picture out of a number of distinct pictures.
This is done several times in succession.
Also, the pictures are presented in a somewhat abstracted version such that the user that set the key can recognize the picture but an adversary has a harder time.
The amount of challenges and the amount of options determine the security of this scheme.
Furthermore there must be a very large set of pictures to choose from.

\subsubsection{TAPI}
\label{sssec:tapi}

TAPI is very similar to a regular PIN entry setup.
It consists of a grid of 16 icons where each icon is split into four parts, this essentially gives you a PIN number pad with 64 possibilities as opposed to just ten.

\subsubsection{CCP}
\label{sssec:ccp}

CCP entails presenting the user with subsequent pictures where the user must click on a certain point in said image.
Depending on the particular point being clicked on we get a different subsequent picture.
The password then is just a sequence of pictures with their respective correct click points.
If the user clicks on the wrong point a different image is presented than expected and thus this serves as a kind of feedback to the user.

\subsubsection{MIBA}
\label{sssec:miba}

This scheme is very similar to CCP but with the addition of several new features.
For one, it has a grid overlay to help the user with selecting click points.
Also, the differentiation between long and short presses on a given point can lead to different passwords in this scheme which aids in improving the password space.

\section{Assessment of the state-of-the-art}
\label{sec:assessment_of_the_state_of_the_art}

In this section we summarize our assessments of the security and usability of different pairing methods.

\subsection{Security assessment}
\label{ssec:security_assessment}

TODO: Add assessments of different security properties.

In the literature there is no known set of criteria by which to evaluate secure pairing methods.
Secure pairing methods are most at risk from a Man-in-the-middle (MITM) attack, whereby an observer may actively or passively intercept (or possibly relay and modify) the communication between devices, which would defeat the desired goal of establishing a secure channel.
This risk, in our view, can be directly quantified in terms of the leakage of communication channels used.

Our considerations for these assessments are as follows:
%\begin{itemize}
The shaking of two phones can incur subtle signals to the accelerometers of the mobile devices, which is hard to observe without having access to the accelerometer data of a device that went through the same shaking pattern.
Using a high-speed camera or similar visual observation of the pairing process does not reveal these subtle signals to an observer.
Graphical passwords may be observed by an outside attacker, the different methods within this category may be vulnerable to a straightforward replay attack while others may be randomized and require several observations before the entire secret is recovered.
Image comparison is vulnerable to the same kind of observation, but the observed information may only be used in one and the same session and is thus not transferable to a straightforward replay attack.
Blinking lights are similarly vulnerable to eavesdropping by an observer, since the light emitted by LED diodes is very bright in a certain light spectrum band, which can be clearly observed at a distance.
Infrared LEDs exhibit the same problem to a worse degree.
HAPADEP makes use of audio signals, and is therefore susceptible to eavesdropping by conventional surveillance methods.
Observing a Seeing is Believing pairing may similarly reveal the secret information that is communicated, but depending on whether this is user-controlled or device-controlled, these signals may transmit information with a comparatively high bandwidth, so that observation is the least practical from the class of pairing methods susceptible to observation.

In conclusion, we propose the following leakage assessment.

\begin{tabular}{c|c}
	Technique & Leakage \\
	\hline
	Shake well before use & Low \\
	Seeing is Believing & Low \\
	Graphical passwords & Low \\
	Image comparison & Medium \\
     Blinking Lights & Medium \\
     HAPADEP & Medium \\
	Infrared & High \\
\end{tabular}


%\end{itemize}

Another factor at play in the security of pairing in practice is the length of the secret message that is agreed upon through the chosen channel, i.e. \emph{key length}.
Since none of the techniques considered define a required minimal or maximal key length, this aspect does not relate to the pairing methods investigated.
In effect, we are free to choose the strength of the used key independently of the chosen method of pairing.

\subsection{Usability assessment}
\label{ssec:usability_assessment}

TODO: Extend explanations and add more different usability properties.

Prior research into the usability of the secure pairing methods found in the literature~\cite{kumar2009comparative} established the System Usability Score (SUS) for the considered pairing methods, as well as measuring the perceived security and the time and success rate to complete the task of pairing by observing a group of 22 test subjects interacting with test impelementations in a controlled environment.
The results of this study show that of the methods considered in this summary, PIN and image comparison methods have the best scores in every category.
Blinking lights takes the longest time to complete, has a relatively low SUS score and the lowest task completion rate.
Seeing is Believing a relatively high task completion rate but on average takes a long time to complete.
The established SUS score is about 50\% and the perceived security is about 60\%.
Shake well before use was not considered in this usability study.

In a similar study \cite{kobsa2009serial} some of the same authors identify the resistance of pairing methods against so-called \textit{rushing user} behaviour.
They identify the distinction between Device Controlled (DC) and User Controlled (UC) pairing methods and observe that the UC pairing methods are more succeptible to the \textit{rushing user} behavior.

\section{Design}
\label{sec:design}

TODO: We do not have an implementation just yet.
We will sketch a preliminary design for a Seeing is Believing (SIB) secure pairing solution for Android mobile devices.
The proposed design will make use of 2D bar codes (specifically, QR codes) to transmit and receive information, and to this end the open source ZXING or ZBar Java libraries will be likely used.
This communication channel will be used to do a Diffie-Hellman (DH) key exchange, which is converted to a PIN code to establish a conventional PIN-based Bluetooth connection pairing.
The conversion will use a custom hash function that uses the binary data that was agreed upon out of band in a regular Simple Secure Pairing (SSP) for Bluetooth 2.1 and higher, or used to derive a PIN code for legacy Bluetooth device pairing.

\subsection{Overview}
\label{ssec:overview}

TODO: We need an implementation first.

\section{Conclusion}
\label{sec:conclusion}

TODO: To be done after the implementation section is finished.

\bibliographystyle{plain}
\bibliography{../references/references.bib}

\end{document}

% vim:tw=0:wrap
